\documentclass{article}
\usepackage{graphicx}
\usepackage{geometry}
\usepackage{amsmath}
\usepackage{amssymb}

\geometry{margin=0.8in}

\title{THU-DiscreteMathmatics-Homework-5}
\author{He Yuhui\quad 2022012050}
\date{Octorber 2024}

\begin{document}

\maketitle

\section{Symbolize}

\begin{itemize}
    \item [1] Use $P_1$ and $P_2$ to denote the two points on the plane, predicate $\varphi(P,l)$ to denote point $P$ on the line $l$, and predicate $\psi(l_1, l_2)$ to denote $l_1$ and $l_2$ is the same line:

        $$
            (\exists l)((\varphi(P_1,l)\land\varphi(P_2,l))\land(\forall l^\prime)((\varphi(P_1, l^\prime)\land\varphi(P_2,l^\prime))\to\psi(l,l^\prime))
        $$

    \item [2] Use predicate $\varphi(x)$ to denote $x\in\mathbb{R}$, $\alpha(x,y),\beta(x,y),\gamma(x,y)$ to denote $x<y,x=y,x>y$:
    
        $$
            (\forall x)(\forall y)((\varphi(x)\land\varphi(y))\to(\alpha(x,y)\lor\beta(x,y)\lor\gamma(x,y)))
        $$

    \item [3] Use predicate people $\varphi(x)$ to denote $x$ is working in Beijing, predicate $\psi(x)$ to denote the household of $x$ is Beijing:
    
        $$
            (\exists x)(\varphi(x)\land\lnot\psi(x))
        $$

    \item [4] Use predicate $\varphi(x)$ to denote $x$ is a kind of metal, predicate $\psi(x)$ to denote $x$ is a kind of liquid, and predicate $\epsilon(x, y)$ to denote $x$ can be dissolved in $y$:

        $$
            (\forall x)(\exists y)((\varphi(x)\land\psi(y))\to\epsilon(x,y))
        $$
\end{itemize}

\section{Translate to natural language}

\begin{itemize}
    \item [1] For all positive integer $x$, $x$ is a rational number and it is also a real number.
    \item [2] For all positive integer $x$, $x$ is a rational number, bot not all rational number is positive interger.
\end{itemize}

\section{Translate to propositional logic formula}

\begin{itemize}
    \item [1] 

        $$
        \begin{aligned}
            &((P(a,a)\to Q(a,a))\lor(P(a,b)\to Q(a,b))\lor(P(a,c)\to Q(a,c)))\\
            \land&((P(b,a)\to Q(b,a))\lor(P(b,b)\to Q(b,b))\lor(P(b,c)\to Q(b,c)))\\
            \land&((P(c,a)\to Q(c,a))\lor(P(c,b)\to Q(c,b))\lor(P(c,c)\to Q(c,c)))
        \end{aligned}
        $$

    \item [2]
    
        $$
        \begin{aligned}
            P(a,a)\lor P(a,b)\lor P(a,c)\lor P(b,a)\lor P(b,b)\lor P(b,c)\lor P(c,a)\lor P(c,b)\lor P(c,c)
        \end{aligned}
        $$

    \item [3]
    
        $$
        \begin{aligned}
            &((P(a,a)\lor P(b,a)\lor P(c,a))\to(Q(a,a)\land Q(b,a)\land Q(c,a)))\\
            \land&((P(a,b)\lor P(b,b)\lor P(c,b))\to(Q(a,b)\land Q(b,b)\land Q(c,b)))\\
            \land&((P(a,c)\lor P(b,c)\lor P(c,c))\to(Q(a,c)\land Q(b,c)\land Q(c,c)))
        \end{aligned} 
        $$
\end{itemize}
\end{document}