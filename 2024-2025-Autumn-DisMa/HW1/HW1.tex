\documentclass{article}
\usepackage{amsmath}
\usepackage{geometry}

\geometry{margin=1in}

\title{THU-Discrete Mathematics-Homework-1}
\author{He YuHui\quad 2022012050}
\date{13 September 2024}

\begin{document}

\maketitle

\section{Homework}

\subsection{Judge the statements}

\begin{itemize}
    \item $P\to P$
    
    \begin{table}[h!]
        \centering
        \begin{tabular}{c|c}
             $P$ & $\textbf{Value}$ \\ 
             \hline
             $\mathsf{T}$ & $\mathsf{T}$\\
             $\mathsf{F}$ & $\mathsf{T}$
        \end{tabular}
        \caption{The truth table of $P\to P$}
        \label{tab:The Truth Tabel-1}
    \end{table}
    
    So $P\to P$ is a \textbf{tautology}.

    \item $(P\vee Q)\to(P\vee Q)$

    We use $R$ to denote $P\vee Q$, so according to the tautology $P\to P$, we know $(P\vee Q)\to(P\vee Q)$ is also a \textbf{tautology}.

    \item $\lnot((P\vee Q)\to(Q\vee P))$

    \begin{table}[h!]
        \centering
        \begin{tabular}{c|c|c}
             $P$ & $Q$ & \textbf{Value} \\
             \hline
             $\mathsf{T}$ & $\mathsf{T}$ & $\mathsf{F}$\\
             $\mathsf{T}$ & $\mathsf{F}$ & $\mathsf{F}$\\
             $\mathsf{F}$ & $\mathsf{T}$ & $\mathsf{F}$\\
             $\mathsf{F}$ & $\mathsf{F}$ & $\mathsf{F}$
        \end{tabular}
        \caption{The truth table of $\lnot((P\vee Q)\to(Q\vee P))$}
        \label{tab:The Truth Tabel-2}
    \end{table}

    So $\lnot((P\vee Q)\to(Q\vee P))$ is a \textbf{contradiction}.
 
\end{itemize}

\subsection{Judge the propositions}

\begin{itemize}
    \item \textbf{This pot of jasmine is so fragrant} is not a proposition, because this is a \textbf{exclamations}.
    
    \item $12$ \textbf{is a prime number} is a proposition, and its truth value is $\mathsf{F}$.

    \item $x+y=2$ is not a proposition, because we do not know what do $x$ and $y$ denote.
\end{itemize}

\subsection{Use symbols to denote}

\begin{itemize}
    \item[1] Use $P$ to denote \textit{today is cold}, $Q$ to denote \textit{it is snowing}:
        \begin{itemize}
            \item[a] $P\to Q$
            \item[b] $P\leftrightarrow Q$
            \item[c] $Q\to P$
        \end{itemize}

    \item[2] Use $P$ to denote \textit{today is cold}, $Q$ to denote \textit{it is snowing}:
        \begin{itemize}
            \item[a] Either it is not cold today, or it is not snowing.
            \item[b] Either it is cold today, or it is not snowing. 
        \end{itemize}
\end{itemize}

\subsection{Answer}

\begin{itemize}
    \item[1] I guess when $P=\mathsf{F}$ and $Q=\mathsf{T}$, the value of entire statement is $\mathsf{T}$.
    \item[2] The truth table is:

    \begin{table}[h!]
        \centering
        \begin{tabular}{c|c|c}
             $P$ & $Q$ & \textbf{Value} \\
             \hline
             $\mathsf{T}$ & $\mathsf{T}$ & $\mathsf{F}$\\
             $\mathsf{T}$ & $\mathsf{F}$ & $\mathsf{F}$\\
             $\mathsf{F}$ & $\mathsf{T}$ & $\mathsf{T}$\\
             $\mathsf{F}$ & $\mathsf{F}$ & $\mathsf{F}$
        \end{tabular}
        \caption{The truth table of $(P\to Q)\wedge\lnot(P\leftrightarrow Q)$}
        \label{tab:The Truth Tabel-2}
    \end{table}
\end{itemize}

\subsection{Formalize the following natural statements}

Use $P$ to denote \textit{He is tall}, $Q$ to denote \textit{He is fat}

\begin{itemize}
    \item[1] $P\wedge Q$
    \item[2] $P\wedge\lnot Q$
    \item[3] $\lnot(P\vee Q)$
    \item[4] $\lnot P\wedge\lnot Q$
\end{itemize}

\end{document}
