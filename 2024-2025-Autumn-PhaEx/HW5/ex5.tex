\documentclass[UTF8]{ctexart}

\usepackage{ctex}
\CTEXsetup[format={\Large\bfseries}]{section}
\usepackage[top=28mm,bottom=28mm,left=15mm,right=15mm]{geometry}

\usepackage{fancyhdr}
\fancypagestyle{plain}{\pagestyle{fancy}}
\pagestyle{fancy}
\lhead{\kaishu 清华大学药学院药理毒理实验}
\newcommand{\numOfReport}[1]{\rhead{\kaishu 实验报告#1}}

\usepackage{fontspec}
\usepackage{wasysym}
\setCJKmainfont[AutoFakeBold={2}]{STZhongsong}
\setCJKmonofont{STZhongsong}

\usepackage{float}
\usepackage{booktabs}
\usepackage{tabularx}
\usepackage{array}
\usepackage{amsmath}
\usepackage{amsfonts}
\usepackage{amssymb}
\usepackage[figuresleft]{rotating}
\usepackage[para]{threeparttable}
\newcommand\info[2][40mm]{\underline{\makebox[#1][c]{#2}}}
\newcommand{\infoTable}[7]{
    \renewcommand\arraystretch{1.4}
    \begin{table}
        \begin{tabularx}{\textwidth}{
        >{\hsize=0.6\hsize\linewidth=\hsize}X
        >{\hsize=0.6\hsize\linewidth=\hsize}X
        >{\hsize=2.0\hsize\linewidth=\hsize}X
        >{\hsize=0.8\hsize\linewidth=\hsize}X
        }
            天气:\info[14mm]{#1} & 温度:\info[14mm]{#2 $^{\circ}\text{C}$} & 湿度:\info[14mm]{#3 $\%$} & 日期:#4\\
            姓名:\info[14mm]{#5} & 班级:\info[14mm]{#6} & 同组人:\info[70mm]{#7} & 
        \end{tabularx}
    \end{table}
}
\newcommand\columnC{\centering\arraybackslash}
\newcommand\columnL{\raggedright\arraybackslash}
\newcommand\columnR{\raggedleft\arraybackslash}

\usepackage{svg}
\usepackage{pdfpages}
\title{阿托品对乙酰胆碱的竞争性拮抗作用和 $\text{pA}_2$ 的测定}
\author{}
\numOfReport{五}

\begin{document}
\infoTable{晴}{13}{70}{10/23/2024}{何昱晖}{药3}{荣子健、马逸然、赵方一澜}
\date{}
\maketitle

\section{实验的目的和原理}

\subsection{实验目的}

\begin{itemize}
    \item [(1)] 观察阿托品对乙酰胆碱的竞争性拮抗作用;
    \item [(2)] 学习受体激动剂的量效关系曲线的绘制方法;
    \item [(3)] 掌握受体激动剂 $\text{pD}_2$ 和受体拮抗剂 $\text{pA}_2$ 的测定的方法和意义
\end{itemize}

\subsection{实验原理}

$\text{pD}_2$ 是评价受体激动剂的效应强度的指标,其定义为:能引起最大效应的 $50\%$ 时的药物剂量的\textbf{摩尔浓度负对数}。$\text{pD}_2=-\lg K_d$(其中 $K_d$ 是药物的解离常数)。其计算方法为:以 $E_x/E_{\max}$ 为纵坐标(其中 $E_x$ 是药物效应,$E_{\max}$ 是药物的最大效应,$\lg C$ 为横坐标作图,得到 S 型曲线,其中间部分为一条直线,计算纵坐标为 $0.5$ 时的横坐标值为 $\lg C_0$,取负数为 $\text{pD}_2$。

$\text{pA}_2$ 是一种用于来表示竞争性拮抗作用的强度指标,其意义是能使激动剂浓度提高到原来 2 倍时,可产生与原来浓度相同效应所需的拮抗剂浓度的负对数,$\text{pA}_2$ 值越大,说明拮抗剂的作用越强。其计算方法为:加入某定量的抑制剂后,再加入上述剂量的激动剂,以 $E_x/E_{\max}$ 为纵坐标,$\lg C$ 为横坐标作图,得到一条拟合直线,纵坐标为 $0.5$ 时,计算得到的横坐标值为 $\lg C_1$。

$$
    \text{pA}_2=\text{pA}_x+\lg\left(\frac{C_1}{C_0}-1\right)
$$

其中:

\begin{itemize}
    \item $\text{pA}_x$ 为拮抗剂摩尔浓度的负对数;
    \item $C_1$ 为加入拮抗剂后引起 $50\%$ 效应的激动剂摩尔浓度;
    \item $C_0$ 为引起 $50\%$ 效应的激动剂摩尔浓度。
\end{itemize}

公式推导过程:

根据占领模型,即激动剂占领受体的比例即为药效与最大药效的比例:

$$
    \frac{[\text{DR}]}{\text{R}_0}=\frac{[\text{D}]}{[\text{D}]+K_d}\Rightarrow C_0=K_d
$$

加入阿托品后可以计算 $C_1$:

$$
    \frac{[\text{DR}]}{[\text{R}_0]}=\dfrac{[\text{D}]}{\text{D}+K_d\left(1+\dfrac{[\text{A}_x]}{K_a}\right)}=\frac{1}{2}\Rightarrow\dfrac{C_1}{C_1+C_0\left(1+\dfrac{[\text{A}_x]}{K_a}\right)}=\frac{1}{2}\Rightarrow\frac{[\text{A}_x]}{K_a}=\frac{C_1}{C_0}-1
$$

又 $\text{pA}_2$ 定义为当激动剂浓度提高到原来的 2 倍时,可产生与原来浓度相同效应所需的拮抗剂浓度的负对数,得:

$$
    \frac{[\text{DR}]}{[\text{R}_0]}=\dfrac{2K_d}{2K_d+K_d\left(1+\dfrac{[\text{A}_2]}{K_a}\right)}=\frac{1}{2}\Rightarrow [\text{A}_2]=K_a
$$

因此

$$
    \frac{[\text{A}_x]}{[\text{A}_2]}=\frac{C_1}{C_0}-1\Rightarrow\text{pA}_2=\text{pA}_x+\lg\left(\frac{C_1}{C_0}-1\right)
$$

\section{实验材料}

\begin{itemize}
    \item 实验动物:豚鼠 1 只,$350\sim 500\text{g}$;
    \item 药品和试剂:阿托品($1\times10^{-7}\text{mol}/\text{L}$)、乙酰胆碱($10^{-2}$、$10^{-3}$、$10^{-4}$、$10^{-5}$、$10^{-6}$、$10^{-7}\text{mol}/\text{L}$)、台氏液、氧气等;
    \item 实验器材:离体组织灌流装置、麦氏浴槽、张力换能器、PowerLab 数据采集系统、注射器、外科手术器械等。
\end{itemize}

\section{实验方法}

\subsection{实验流程}

\begin{itemize}
    \item [1] PowerLab 仪器参数设置:使用「张力测定实验.adiset」文件,测试仪器使用 1 通道,设置参数采样速率为 $200/\text{s}$,量程 $10\text{g}$,调零;
    \item [2] 台氏液每组 $300\text{mL}$,$37^\circ\text{C}$ 保温,取 $20\text{mL}$ 用氧气饱和备用;
    \item [3] 麦氏浴槽中加 $20\text{mL}$ 台氏液,调节温度 $37\pm1^\circ\text{C}$,氧气 $2\sim 3$ 个气泡/秒;
    \item [4] 豚鼠回肠标本制备:取豚鼠,用木棒猛击头部处死,迅速解剖腹腔,找到回肠,剪取 20cm 左右回肠置于氧饱和台氏液中,剥离脂肪,用眼科剪剪成 $2\sim 2.5\text{cm}$ 左右的肠段,冲洗干净内容物。将肠管标本两端用缝针穿线,打结固定;一端打空结(约 $1\text{cm}$ 左右),另一端穿长线打结,用眼科镊钳住空结固定于弯钩上,放入麦氏浴槽,固定弯钩;将另一端长线与张力换能器相连;
    \item [5] 调整张力换能器高度,使得前负荷为 $5\sim 7\text{g}$ 左右;回肠标本在浴槽内平衡,每 10min 换液一次,共换液 3 次;
    \item [6] 最后一次换液后,浴槽中加入台氏液 20mL 平衡 10min,秒级一段正常曲线,随后按照表 1 次序,小剂量连续加入乙酰胆碱(Ach),制作 Ach 的累积效应曲线。每加一次药时都需标记,指标加药序号即可,直至曲线上升至最高峰不再升高为止。每次加样前,用移液枪先取出槽内相应体积的溶液,再加入药液;
    \item [7] 放掉麦氏浴槽中的溶液,用新鲜台氏液冲洗肠管 3 次,每次平衡 10 min,稳定标本共 30min,随后加入阿托品 0.2 mL(终浓度位 $10^{-9}\text{mol}/\text{L}$),1min 后加入乙酰胆碱,按照上述方法制作 Ach 累积效应曲线;
    \item [8] 实验结束,清洗所用容器和麦氏浴槽,关闭仪器。
\end{itemize}

\begin{table}[H]
    \centering
    \begin{threeparttable}[b]
        \caption{乙酰胆碱加样浓度和加样量}
        \quad

        \begin{tabularx}{\textwidth}{
            >{\columnC\hsize=0.6\hsize\linewidth=\hsize}X
            >{\columnC\hsize=1\hsize\linewidth=\hsize}X
            >{\columnC\hsize=1\hsize\linewidth=\hsize}X
            >{\columnC\hsize=1.4\hsize\linewidth=\hsize}X
        }
            \toprule[1.5pt]
            加药次序 & Ach 浓度($\text{mol}/\text{L}$) & 加入量(mL) & 浴槽中 Ach 累计浓度($\mu\text{mol}/\text{L}$)\\
            1 & $10^{-7}$ & 0.2 & 0.001\\
            2 & $10^{-7}$ & 0.4 & 0.003\\
            3 & $10^{-6}$ & 0.14 & 0.01\\
            4 & $10^{-6}$ & 0.4 & 0.03\\
            5 & $10^{-5}$ & 0.14 & 0.1\\
            6 & $10^{-5}$ & 0.4 & 0.3\\
            7 & $10^{-4}$ & 0.14 & 1\\
            8 & $10^{-4}$ & 0.4 & 3\\
            9 & $10^{-3}$ & 0.14 & 10\\
            10 & $10^{-3}$ & 0.4 & 30\\
            11 & $10^{-2}$ & 0.14 & 100\\
            12 & $10^{-2}$ & 0.4 & 300\\
            \bottomrule[1.5pt]
        \end{tabularx}
    \end{threeparttable}
\end{table}

\subsection{注意事项}

\begin{itemize}
    \item [1] 处死豚鼠取肠要迅速、轻巧,并置于氧饱和台氏液中保持活性;
    \item [2] 给予回肠肠管的前负荷不能太大(最开始时为 $5\sim 7\text{g}$,随着孵育时间的延长,其张力会降低);
    \item [3] 每次给予 Ach 时需等到前一次反应达到最大值时才能给下一个剂量的药物,即描记曲线到达平台阶段。制作整个累积量效曲线过程中不能换溶液;
    \item [4] 切勿随意更改生物信号处理采集系统的实验参数设置。
\end{itemize}

\section{实验结果}

图1是未加阿托品时测定的乙酰胆碱累积效应曲线图,平衡时的负荷约为 $4.5\text{g}$,最高值约为 $7\text{g}$:

\begin{figure}[H]
    \centering
    \includegraphics[width=12.5cm]{image-1.png}
    \caption{未加阿托品时的乙酰胆碱累积效应曲线}
\end{figure}

图2是加入阿托品时测定的乙酰胆碱累积效应曲线图,平衡时的负荷约为 $5.25\text{g}$,最高值约为 $7\text{g}$:

\begin{figure}[H]
    \centering
    \includegraphics[width=12.5cm]{image-2.png}
    \caption{加入阿托品时的乙酰胆碱累积效应曲线}
\end{figure}

表2是测定曲线中各段的平均值对比:

\begin{table}[H]
    \centering
    \begin{threeparttable}[b]
        \caption{对照组与实验组各阶段平均值比较}
        \quad

        \begin{tabularx}{\textwidth}{
            >{\columnC\hsize=1\hsize\linewidth=\hsize}X
            >{\columnC\hsize=1\hsize\linewidth=\hsize}X
            >{\columnC\hsize=1\hsize\linewidth=\hsize}X
        }
            \toprule[1.5pt]
            阶段 & 对照组 & 实验组 \\
            \midrule
            校准平衡 & 4.5377 & 5.2161\\
            加入阿托品 & / & 5.1461\\
            1 & 4.535 & 5.1459\\
            2 & 4.563 & 5.1417\\
            3 & 4.6307 & 5.1334\\
            4 & 4.9187 & 5.1393\\
            5 & 5.3415 & 5.2531\\
            6 & 5.8949 & 5.5183\\
            7 & 6.1906 & 6.0077\\
            8 & 6.4545 & 6.4767\\
            9 & 6.4946 & 6.6281\\
            10 & 6.5708 & 6.5234\\
            11 & 6.7552 & 6.5532\\
            12 & 6.7491 & 6.8002\\
            \bottomrule[1.5pt]
        \end{tabularx}
    \end{threeparttable}
\end{table}

\section{课后思考题}


\end{document}