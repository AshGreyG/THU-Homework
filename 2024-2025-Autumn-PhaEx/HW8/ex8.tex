\documentclass[UTF8]{ctexart}
\usepackage{graphicx}

\usepackage{ctex}
\CTEXsetup[format={\Large\bfseries}]{section}
\usepackage[top=28mm,bottom=28mm,left=15mm,right=15mm]{geometry}

\usepackage{fancyhdr}
\fancypagestyle{plain}{\pagestyle{fancy}}
\pagestyle{fancy}
\lhead{\kaishu 清华大学药学院药理毒理实验}
\newcommand{\numOfReport}[1]{\rhead{\kaishu 实验报告#1}}

\usepackage{fontspec}
\usepackage{wasysym}
\setCJKmainfont[AutoFakeBold={2}]{STZhongsong}
\setCJKmonofont{STZhongsong}

\usepackage{float}
\usepackage{booktabs}
\usepackage{tabularx}
\usepackage{array}
\usepackage{amsmath}
\usepackage{amsfonts}
\usepackage{amssymb}
\usepackage[figuresleft]{rotating}
\usepackage[para]{threeparttable}
\newcommand\info[2][40mm]{\underline{\makebox[#1][c]{#2}}}
\newcommand{\infoTable}[7]{
    \renewcommand\arraystretch{1.4}
    \begin{table}
        \begin{tabularx}{\textwidth}{
        >{\hsize=0.6\hsize\linewidth=\hsize}X
        >{\hsize=0.6\hsize\linewidth=\hsize}X
        >{\hsize=2.0\hsize\linewidth=\hsize}X
        >{\hsize=0.8\hsize\linewidth=\hsize}X
        }
            天气:\info[14mm]{#1} & 温度:\info[14mm]{#2 $^{\circ}\text{C}$} & 湿度:\info[14mm]{#3 $\%$} & 日期:#4\\
            姓名:\info[14mm]{#5} & 班级:\info[14mm]{#6} & 同组人:\info[70mm]{#7} & 
        \end{tabularx}
    \end{table}
}
\newcommand\columnC{\centering\arraybackslash}
\newcommand\columnL{\raggedright\arraybackslash}
\newcommand\columnR{\raggedleft\arraybackslash}

\usepackage{svg}
\usepackage{pdfpages}

\title{硫酸链霉素的毒性反应及氯化钙的拮抗作用}
\author{}
\numOfReport{八}

\begin{document}

\infoTable{小雨}{10}{83}{11/06/2024}{何昱晖}{药3}{荣子健、马逸然、赵方一澜}
\date{}
\maketitle

\section{实验目的及原理}

\subsection{实验目的}

\begin{itemize}
    \item [1] 观察链霉素引起的小鼠急性毒性症状;
    \item [2] 观察氯化钙对链霉素中毒小鼠的保护作用,了解链霉素中毒的解救方法。
\end{itemize}

\subsection{实验原理}

氨基糖苷类抗生素的主要不良反应有耳毒性、肾毒性、过敏反应和骨骼肌收缩无力。链霉素为氨基糖苷类药物,在大剂量静脉滴注或腹腔注射时,其与血液中的钙离子络合,体内游离的钙离子浓度下降,抑制了钙离子参与的 Ach 的释放,出现四肢软弱无力、呼吸困难,甚至呼吸停止等毒性反应。给予钙制剂后,细胞外钙离子浓度升高,可竞争性的拮抗氨基糖苷类药物对钙离子通道的阻碍作用,恢复神经末梢正常的钙离子内流,使 Ach 的释放增多,即可解除肌无力的症状。临床上一般使用葡萄糖酸钙和新斯的明抢救。

\section{实验材料}

\begin{itemize}
    \item 实验动物:小鼠,雄性,体重 $18\sim 22\text{g}$;
    \item 药品和试剂:$4\%$ 硫酸链霉素溶液、$1\%$ 氯化钙溶液、生理盐水;
    \item 实验器材:电子秤、计时器、1mL 注射器、小鼠尾静脉注射显像仪、小鼠抓力仪。
\end{itemize}

\section{实验方法}

\begin{itemize}
    \item [1] 每组取 4 只小鼠,实验前称重、编号,随机分为实验组和对照组,各 2 只小鼠。观察其正常的呼吸、活动情况、肌紧张程度。采用小鼠抓力仪测定小鼠的抓力;
    \item [2] 实验组和对照组小鼠均腹腔注射 $4\%$ 硫酸链霉素溶液 0.1mL/10g,注射后开始计时,观察并记录小鼠的呼吸、活动情况和肌紧张程度的变化。硫酸链霉素给药后 $5\sim 8\text{min}$,会观察到小鼠出现明显肌无力现象,此时采用小鼠抓力仪测定小鼠的抓力;
    \item [3] 抓力测定结束后,立刻对小鼠进行解救。实验组小鼠尾静脉缓慢推注 $1\%$ 氯化钙溶液 0.1mL/10g,对照组小鼠尾静脉注射生理盐水 0.1ml/10g,注射后观察并记录两组小鼠的呼吸、活动情况和肌肉紧张程度的变化,$10\sim 15\text{min}$ 后再次用小鼠抓力仪测定小鼠的抓力。
\end{itemize}

注意事项:

\begin{itemize}
    \item [1] $1\%$ 1氯化钙溶液尾静脉注射要缓慢,因钙盐而兴奋心脏,注射过快会使血钙浓度突然增高,引起小鼠心率失常,甚至心脏骤停;
    \item [2] 采用抓力测定仪测定小鼠抓力的时候,平行测定 3 次,每次测试的时间不宜超过 1 分钟,以避免疲劳和压力对测试结果的影响。在测试过程中,对小鼠需要使用较轻的力量来拉动测试装置,以避免对小鼠造成伤害;
    \item [3] 也可以采用小鼠倒抓鼠笼盖的方法来评价小鼠肌张力。硫酸链霉素注射后,小鼠肌张力降低,小鼠无力倒抓鼠笼盖;尾静脉推注氯化钙后,可采用此法,简单评价小鼠肌张力的恢复时间(实现倒抓 10 秒以上视为肌张力恢复)。
\end{itemize}

\section{实验结果}

\begin{table}[H]
    \centering
    \begin{threeparttable}[b]
        \caption{硫酸链霉素对小鼠的毒性作用及解救}
        \quad

        \begin{tabularx}{\textwidth}{
            >{\columnC\hsize=1\hsize\linewidth=\hsize}X
            >{\columnC\hsize=1\hsize\linewidth=\hsize}X
            >{\columnC\hsize=1\hsize\linewidth=\hsize}X
            >{\columnC\hsize=1\hsize\linewidth=\hsize}X
            >{\columnC\hsize=1\hsize\linewidth=\hsize}X
            >{\columnC\hsize=1\hsize\linewidth=\hsize}X
            >{\columnC\hsize=1\hsize\linewidth=\hsize}X
            >{\columnC\hsize=1\hsize\linewidth=\hsize}X
            >{\columnC\hsize=1\hsize\linewidth=\hsize}X
        }
        \toprule[1.5pt]
        小鼠编号 & 给药前抓力 & 给药前呼吸/活动情况 & 腹腔注射药物 & 链霉素给药后抓力 & 链霉素给药后呼吸/活动情况 & 尾静脉注射药物 & 尾静脉注射后抓力 & 尾静脉注射后呼吸/活动情况\\
        \midrule
        1 & 1.217 & 正常 & 链霉素 & 给药后约4min出现静趴、半闭眼,四肢摊开并不再反抗触摸 & 0.351 & 氯化钙 & / & 0.898\\
        \midrule
        2 & 1.230 & 正常 & 链霉素 & $\sim$ & 0.897 & 氯化钙 & / & 1.143\\
        \midrule
        3 & 1.218 & 正常 & 链霉素 & $\sim$ & 0.926 & 生理盐水 & / & 0.919\\
        \midrule
        4 & 1.368 & 正常 & 链霉素 & $\sim$ & 1.102 & 生理盐水 & / & 1.138\\
        \bottomrule[1.5pt]
        \end{tabularx}
    \end{threeparttable}
\end{table}

\section{课后思考题}

m\begin{itemize}
    \item [1] 链霉素的不良反应有哪些?钙盐可防治链霉素的哪些毒性反应?

        不良反应主要有耳毒性、肾毒性以及对神经肌肉的阻滞作用,也可引起过敏反应;钙盐可以防治链霉素对神经肌肉的阻滞作用;

    \item [2] 链霉素中毒选择哪些药物进行解救?机制是什么?

        葡萄糖酸钙或氯化钙的机理是与链霉素结合,阻止链霉素与突触前膜上的钙结合部位结合,从而阻断乙酰胆碱的释放,而使中毒现象减轻或消失;新斯的明抑制胆碱酯酶,增强乙酰胆碱的作用。

\end{itemize}

\end{document}
