\documentclass[UTF8]{ctexart}

\usepackage{ctex}
\CTEXsetup[format={\Large\bfseries}]{section}
\usepackage[top=28mm,bottom=28mm,left=15mm,right=15mm]{geometry}

\usepackage{fancyhdr}
\fancypagestyle{plain}{\pagestyle{fancy}}
\pagestyle{fancy}
\lhead{\kaishu 清华大学药学院药理毒理实验}
\newcommand{\numOfReport}[1]{\rhead{\kaishu 实验报告#1}}

\usepackage{fontspec}
\usepackage{wasysym}
\setCJKmainfont[AutoFakeBold={2}]{STZhongsong}
\setCJKmonofont{STZhongsong}

\usepackage{float}
\usepackage{booktabs}
\usepackage{tabularx}
\usepackage{array}
\usepackage{amsmath}
\usepackage{amsfonts}
\usepackage{amssymb}
\usepackage[figuresleft]{rotating}
\usepackage[para]{threeparttable}
\newcommand\info[2][40mm]{\underline{\makebox[#1][c]{#2}}}
\newcommand{\infoTable}[7]{
    \renewcommand\arraystretch{1.4}
    \begin{table}
        \begin{tabularx}{\textwidth}{
        >{\hsize=0.6\hsize\linewidth=\hsize}X
        >{\hsize=0.6\hsize\linewidth=\hsize}X
        >{\hsize=2.0\hsize\linewidth=\hsize}X
        >{\hsize=0.8\hsize\linewidth=\hsize}X
        }
            天气:\info[14mm]{#1} & 温度:\info[14mm]{#2 $^{\circ}\text{C}$} & 湿度:\info[14mm]{#3 $\%$} & 日期:#4\\
            姓名:\info[14mm]{#5} & 班级:\info[14mm]{#6} & 同组人:\info[70mm]{#7} & 
        \end{tabularx}
    \end{table}
}
\newcommand\columnC{\centering\arraybackslash}
\newcommand\columnL{\raggedright\arraybackslash}
\newcommand\columnR{\raggedleft\arraybackslash}

\usepackage{svg}
\usepackage{pdfpages}
\title{不同给药途径对硫酸镁药理作用的影响}
\author{}
\numOfReport{三}

\begin{document}
\infoTable{晴}{23}{80}{10/9/2024}{何昱晖}{药3}{荣子健、马逸然、赵方一澜}
\date{}
\maketitle

\section{实验目的和原理}

\subsection{实验目的}

\begin{itemize}
    \item [(1)] 了解影响药物疗效的各种因素,了解给药途径对药物疗效的影响;
    \item [(2)] 观察不同给药途径对硫酸镁药理作用的影响。
\end{itemize}

\subsection{实验原理}

药物疗效受到多种因素的影响,其中给药途径不同,不仅影响到药物作用的快慢、强弱以及维持时间的长短,有时还会产生不同的药理作用。硫酸镁口服基本不吸收而产生导泻作用,而注射给药由于大量镁离子进入血液具有降低血压,抗惊厥和肌肉松弛的作用,其作用机理为拮抗钙离子,舒张平滑肌。

\section{实验材料}

\begin{itemize}
    \item 实验动物:ICR 小鼠,雄性,体重 $24\sim 26\text{kg}$;
    \item 药品和试剂:硫酸镁、活性炭末;
    \item 实验器材:注射器(1.0 mL)、灌胃针、天平、托盘、软尺
\end{itemize}

\section{实验方法}

\begin{itemize}
    \item [(1)] 药物配制:$5\%$ 硫酸镁,$5\%$ 碳末/$5\%$ 阿拉伯胶,$20\%$ 硫酸镁(含 $5\%$ 碳末/$5\%$ 阿拉伯胶):以上溶液必须在生理盐水中配制;
    \item [(2)] 实验分组:各小组分别取小鼠 12 只,称重,编号。小鼠预先禁食 8 hr;
    \item [(3)] 灌胃给药:取已称重、标记的 6 只小鼠,其中 3 只小鼠灌胃 $5\%$ 炭末/$5\%$ 阿拉伯胶($0.1\text{mL}/10\text{g}$),另 3 只小鼠灌胃 $20\%$ 硫酸镁(含 $5\%$ 炭末加 $5\%$ 阿拉伯胶;$0.1\text{mL}/10\text{g}$)。给药后立刻计时,分别于灌胃 15min 后拉脱颈椎处死小鼠,剪取从幽门到盲肠段肠管,除去肠系膜,小心拉直肠管,量取黑色前沿距离幽门的长度以及幽门到盲肠的长度并记录。
    \item [(4)] 注射给药:取已称重、标记的另 6 只小鼠,其中 3 只腹腔注射 $5\%$ 硫酸镁($0.1\text{mL}.10\text{g}$),观察小鼠的行为学改变,包括肌张力、呼吸频率、活动能力等,记录死亡时间。另 3 只尾静脉注射 $5\%$ 硫酸镁($0.1\text{mL}/10\text{g}$),观察小鼠的死亡时间和死亡前行为学的改变。
\end{itemize}

\section{实验结果}

\begin{table}[H]
    \centering
    \begin{threeparttable}[b]
        \caption{$20\%$ 硫酸镁对小鼠小肠推进速度的影响——本组数据}
        \quad
        
        \begin{tabularx}{\textwidth}{
            >{\columnC\hsize=0.4\hsize\linewidth=\hsize}X
            >{\columnC\hsize=0.5\hsize\linewidth=\hsize}X
            >{\columnC\hsize=4\hsize\linewidth=\hsize}X
            >{\columnC\hsize=0.5\hsize\linewidth=\hsize}X
            >{\columnC\hsize=0.5\hsize\linewidth=\hsize}X
            >{\columnC\hsize=0.5\hsize\linewidth=\hsize}X
            >{\columnC\hsize=0.6\hsize\linewidth=\hsize}X
        }
            \toprule[1.5pt]
            序号 & 小鼠体重\tnote{1} & 药物 & 给药量\tnote{2} & 墨汁前沿距离幽门的距离\tnote{3} & 幽门-盲肠距离\tnote{3} & 小肠推进/幽门-盲肠长度百分比\\
            \midrule
            1 & 22 & $5\%$ 炭末/$5\%$ 阿拉伯胶 & 0.22 & 19.5 & 37.6 & $51.86\%$\\
            \midrule
            2 & 23 & $5\%$ 炭末/$5\%$ 阿拉伯胶 & 0.23 & 20.1 & 34.2 & $58.77\%$\\
            \midrule
            3 & 23 & $5\%$ 炭末/$5\%$ 阿拉伯胶 & 0.23 & 18.7 & 39.5 & $47.34\%$\\
            \midrule
            4 & 22 & $20\%$ 硫酸镁(含 $5\%$ 炭末/$5\%$ 阿拉伯胶)& 0.22 & 26 & 32.4 & $80.25\%$\\
            \midrule
            5 & 23 & $20\%$ 硫酸镁(含 $5\%$ 炭末/$5\%$ 阿拉伯胶)& 0.23 & 38.2 & 45 & $84.89\%$\\
            \midrule
            6 & 24 & $20\%$ 硫酸镁(含 $5\%$ 炭末/$5\%$ 阿拉伯胶)& 0.24 & 31.8 & 44.6 & $71.30\%$\\
            \bottomrule[1.5pt]
        \end{tabularx}
        \begin{tablenotes}
            \item [1] 单位 $\text{g}$
            \item [2] 单位 $\text{mL}$
            \item [3] 单位 $\text{cm}$
        \end{tablenotes}
    \end{threeparttable}
\end{table}

\begin{table}[H]
    \centering
    \begin{threeparttable}[b]
        \caption{$5\%$ 硫酸镁注射给药小鼠行为学改变和死亡情况——本组数据}
        \quad

        \begin{tabularx}{\textwidth}{
            >{\columnC\hsize=0.5\hsize\linewidth=\hsize}X
            >{\columnC\hsize=0.5\hsize\linewidth=\hsize}X
            >{\columnC\hsize=1.5\hsize\linewidth=\hsize}X
            >{\columnC\hsize=0.5\hsize\linewidth=\hsize}X
            >{\columnC\hsize=0.5\hsize\linewidth=\hsize}X
            >{\columnC\hsize=0.5\hsize\linewidth=\hsize}X
            >{\columnC\hsize=3\hsize\linewidth=\hsize}X
        }
            \toprule[1.5pt]
            序号 & 小鼠体重\tnote{1} & 给药途径 & 给药量\tnote{2} & 是否死亡 & 死亡时间\tnote{3} & 小鼠行为学改变 \\
            \midrule
            7 & 22 & 腹腔注射 & 0.22 & $-$ & / & 注射后2min尚能沿盒爬动,3min趋于安静,趴卧、尾根变白,约1s/次呼吸,约15min出现失禁现象。大约40min恢复活力,至1小时完全恢复正常爬行状态。\\
            \midrule
            8 & 22 & 腹腔注射 & 0.22 & $-$ & / & 注射后约2min~3min趋于安静,随后尾根和四肢变白,爬行无力,7min尾3/4变白,10min基本全白,约1s/次呼吸。此时可轻松捏背部提起且反抗轻微,肌肉松弛。大约40min恢复活力,至1小时完全恢复正常爬行状态。\\
            \midrule
            9 & 22 & 腹腔注射 & 0.22 & $-$ & / & 1min安静,5min肌无力静趴,呼吸减弱。15min呼吸变慢。\\
            \midrule
            10 & 22 & 尾静脉注射 & 0.22 & $+$ & 10 & 给药短促抽动后立刻死亡\\
            \midrule
            11 & 23 & 尾静脉注射 & 0.23 & $+$ & 20 & 给药短促抽动后立刻死亡\\
            \midrule
            12 & 22 & 尾静脉注射 & 0.23 & $+$ & 20 & 给药短促抽动后立刻死亡\\
            \bottomrule[1.5pt]
        \end{tabularx}
        \begin{tablenotes}
            \item [1] 单位 $\text{g}$
            \item [2] 单位 $\text{mL}$
            \item [3] 单位 $\text{s}$
        \end{tablenotes}
    \end{threeparttable}
\end{table}

\begin{table}[H]
    \centering
    \begin{threeparttable}[b]
        \caption{小鼠灌胃 $20\%$ 硫酸镁 15min 后对小肠推进功能的影响——全班数据}
        \quad

        \begin{tabularx}{\textwidth}{
            >{\columnC\hsize=1\hsize\linewidth=\hsize}X
            >{\columnC\hsize=1\hsize\linewidth=\hsize}X
            >{\columnC\hsize=1\hsize\linewidth=\hsize}X
            >{\columnC\hsize=1\hsize\linewidth=\hsize}X
        }
            \toprule[1.5pt]
            组别 & 动物数 & 小鼠推进长度(cm) & 小肠推进/幽门-盲肠长度百分比\\
            \midrule
            硫酸镁组 & 18 & $33.79\pm 6.33$ & $82.96\%\pm7.13\%$ \\
            \midrule
            对照组 & 15 & $19.89\pm 4.44$ & $58.34\%\pm8.38\%$ \\
            \bottomrule[1.5pt]
        \end{tabularx}
    \end{threeparttable}
\end{table}

\begin{table}[H]
    \centering
    \begin{threeparttable}[b]
        \caption{$5\%$ 硫酸镁注射给药后小鼠行为学改变及死亡情况——全班数据}
        \quad

        \begin{tabularx}{\textwidth}{
            >{\columnC\hsize=0.5\hsize\linewidth=\hsize}X
            >{\columnC\hsize=0.5\hsize\linewidth=\hsize}X
            >{\columnC\hsize=0.5\hsize\linewidth=\hsize}X
            >{\columnC\hsize=0.5\hsize\linewidth=\hsize}X 
            >{\columnC\hsize=2.5\hsize\linewidth=\hsize}X 
        }
            \toprule[1.5pt]
            组别 & 动物数 & 死亡比例 & 平均死亡时间(s) & 小鼠行为学改变情况 \\
            \midrule
            腹腔注射 & 18 & $0.00\%$ & 0 & 给药2min后安静少动,保持缓慢呼吸、安静,约50min后恢复正常\\
            \midrule
            尾静脉注射 & 18 & $100.00\%$ & 8.94 & 给药短促抽动后立刻死亡\\
            \bottomrule[1.5pt]
        \end{tabularx}
    \end{threeparttable}
\end{table}

\section{课后思考题}

\begin{itemize}
    \item [1] 简述硫酸镁不同途径给药时药理作用不同的生理机制,并根据其作用机制,推测临床中出现硫酸镁中毒时的症状并给出解救方案;

        腹腔注射:硫酸镁在肠道中的浓度高过体液浓度,因为渗透压的原因使得身体的水分进入肠道产生泻的作用,大量经口摄入硫酸镁,会导致严重腹泻,可口服抑制肠道蠕动的药物;静脉注射:镁离子可抑制中枢神经的活动,抑制运动神经-肌肉接头乙酰胆碱的释放,阻断神经肌肉联结处的传导,降低或解除肌肉收缩作用,同时对血管平滑肌有舒张作用,使痉挛的外周血管扩张,降低血压,硫酸镁用药过量,应施以人工辅助通气,并缓慢注射钙剂解救。常用的为 $10\%$ 葡萄糖酸钙注射液 10mL 缓慢注射
    
    \item [2] 本实验中产生误差的主要原因有哪些?实验过程中需要注意什么?
     
        主要误差的原因有两点,第一是托盘不够完全展开小肠,需要注意将小肠对折进行测量,不可避免在对折处有测量误差;第二是在解剖出小肠时的摆动可能会导致炭末在肠道中往下滑落,导致测量误差,注意平放肠道剪开肠系膜;

    \item [3] 查阅资料,试从药物的吸收、分布、代谢、排泄方面阐述不同给药途径造成不同药理、药效、作用时间可能存在的机制。

        腹腔注射的硫酸镁经肠壁吸收,渗透压将引起大量的水进入肠壁,产生导泄作用,作用时间较长;静脉注射的硫酸镁进入血液循环,快速抑制中枢神经的活动,药效为降低或解除肌肉收缩作用,作用时间迅速。

\end{itemize}
    
\end{document}